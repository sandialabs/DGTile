\section{Theory}

\subsection{Introduction}

Discontinuous Galerkin finite element basis functions can take many forms.
In particular, these basis functions are often categorized as \emph{modal}
or \emph{nodal}. Modal and nodal DG basis functions of the same polynomial
order typically span the same space, but differ in manners that we will not
elaborate on presently. In DGTile, we provide modal basis functions in
$d=1,2,3$ spatial dimensions of polynomial order $p=0,1,2,3$ on a fixed
reference cell $[-1,1]^d$. These basis functions are constructed as products
of Legendre polynomials and are \emph{orthogonal} over the reference element.

\subsection{Spatial Representation}

\begin{equation}
u(\bs{\xi}, t) = \sum_{e=1}^{n_e} \sum_{a=1}^{n_b} c_a^e(t) \phi_a(\bs{\xi})
\end{equation}

\subsection{Legendre Polynomials}

The first four Legendre polynomials of order $p$ on the one-dimensional
interval $[-1,1]$ are given as
%
\begin{equation}
\begin{aligned}
\mathcal{L}_0(x) &= 1, \\
\mathcal{L}_1(x) &= x, \\
\mathcal{L}_2(x) &= \frac12 (3x^2-1), \\
\mathcal{L}_3(x) &= \frac12 (5x^3-3x).
\end{aligned}
\label{eq:legendre_polynomials}
\end{equation}
%
with derivatives
%
\begin{equation}
\begin{aligned}
\mathcal{L}'_0(x) &= 0, \\
\mathcal{L}'_1(x) &= 1, \\
\mathcal{L}'_2(x) &= 3x, \\
\mathcal{L}'_3(x) &= \frac32(5x^2-1).
\end{aligned}
\label{eq:legendre_polynomial_derivatives}
\end{equation}
%
The Legendre polynomials are \emph{orthogonal} over the interval $[-1,1]$,
where
%
\begin{equation}
\int_{-1}^1 \; \mathcal{L}_m(x) \mathcal{L}_n(x) \, \text{d}x =
\frac{2}{2n + 1} \delta_{mn},
\end{equation}
%
where $\delta_{mn}$ denotes the Kronecker delta (equal to $1$ if $m=n$ and
$0$ otherwise).

\subsection{Tensor Product Bases}

\begin{equation}
\begin{aligned}
\left\{ \phi_a(\bs{\xi}) \right\}_{a=1}^{n_b} &=
  \left\{ \mathcal{L}_i(\xi_1) \; | \; i \in \mathbb{N}_0 \land i \leq p
  \right\}, \\
\left\{ \phi_a(\bs{\xi}) \right\}_{a=1}^{n_b} &=
  \left\{ \mathcal{L}_i(\xi_1) \mathcal{L}_j(\xi_2) \; | \; i,j \in
  \mathbb{N}_0 \land i,j \leq m \right\}, \\
\left\{ \phi_a(\bs{\xi}) \right\}_{a=1}^{n_b} &=
  \left\{ \mathcal{L}_i(\xi_1) \mathcal{L}_j(\xi_2) \mathcal{L}_k(\xi_3) \; |
  \; i,j,k \in \mathbb{N}_0 \land i,j,k \leq p \right\}
\end{aligned}
\label{eq:tensor_basis}
\end{equation}

\subsection{Complete Polynomial Bases}

\begin{equation}
\begin{aligned}
\left\{ \phi_a(\bs{\xi}) \right\}_{a=1}^{n_b} &=
  \left\{ \mathcal{L}_i(\xi_1) \; | \; i \in \mathbb{N}_0 \land i \leq p
  \right\}, \\
\left\{ \phi_a(\bs{\xi}) \right\}_{a=1}^{n_b} &=
  \left\{ \mathcal{L}_i(\xi_1) \mathcal{L}_j(\xi_2) \; | \; i,j \in
  \mathbb{N}_0 \land i + j \leq m \right\}, \\
\left\{ \phi_a(\bs{\xi}) \right\}_{a=1}^{n_b} &=
  \left\{ \mathcal{L}_i(\xi_1) \mathcal{L}_j(\xi_2) \mathcal{L}_k(\xi_3) \; |
  \; i,j,k \in \mathbb{N}_0 \land i + j + k \leq p \right\}
\end{aligned}
\label{eq:complete_basis}
\end{equation}

\subsection{Numerical Integration}

\begin{equation}
\int_{-1}^{1} f(x) \, \text{d}x \approx \sum_{i=1}^{q} w_q f(\xi_q)
\end{equation}
